\section{Walk-Through}
\label{sec:walk-through}

\subsection{Step 1: ESL Analysis and Refactoring}
\label{sec:walk-through-step1}

\begin{lstlisting}[language=C++,
caption={Example of a SystemC module},
label={lst:walk-through},
numbers=left,
captionpos=b,   
basicstyle={\footnotesize},
xleftmargin=5.0ex]
struct Example: public sc_module{
  //Constructor
  Example(sc_module_name name):
          value(9){SC_THREAD(fsm);}
  SC_HAS_PROCESS(Example);

  //Ports
  blocking_in<int> b_in;
  blocking_out<bool> b_out;

  //Variables
  int value;

  //Finite State Machine
  void fsm(){
  while(true){
    b_in->read(value);
    if(value > 10){
      b_out->write(true);
    }else b_out->write(false);
  }}
};
\end{lstlisting}

The PDD-methodology starts with a system-level design serving as golden reference for the following design process. %
Here, SystemC is used as system-level design entry language. %
A system is composed of a set of communicating modules. %
At the system level the main focus is on describing and verifying the \emph{communication} between modules. 
Each of these modules is later transformed into a set of properties. %

Listing~\ref{lst:walk-through} shows the description of a SystemC module, named \textit{Example}, with one input and one output. %
The inputs and outputs are connected with other modules via \emph{channels} describing the underlying communication protocol. %
In this module, the inputs (\textit{b\_in}) and outputs (\textit{b\_out}) use a \textit{blocking} interface. %
Blocking, in this context, means that the underlying communication protocol implements a four-phase handshake. %
Thus, the message is transmitted if and only if both sides are ready for communication. %

The behavior of the module is described within the method \textit{fsm}
that is registered as a thread with the SystemC scheduler and is
executed infinitely often. %
The module reads a value from the input \textit{b\_in} and stores the
result in the variable \textit{value}. %
If \textit{value} is $>$\textit{10} the output port sends
\textit{true}, otherwise \textit{false}. %

As you might already suspect from examining
listing~\ref{lst:walk-through}, it is not possible to use arbitrary
SystemC constructs when following the PDD paradigm. %
Only a certain subset of SystemC is allowed, and the code needs to
follow a certain structure and coding rules. %
The reason is that SystemC, per se, does not have a clear semantics
with respect to cycle-accurate RTL implementation. %
The semantics need to be introduced by restricting the use of language
constructs to what we call the ``designable subset'' of SystemC. %
This subset will be described and detailed in the following. %

Let's continue with our first walk-through of PDD. %
We assume that our SystemC module has passed verification on the ESL
and is considered ready for implementation. %
The SystemC code is analyzed using \DeSCAM{}. %
We run the tool from the command line as \texttt{DeSCAM <path-to-file> -AML}
The following example shows the output of DeSCAM %
when no errors are found:
\begin{small}
\begin{verbatim}
############################
Module: Example
############################

======================
PPA generation:
----------------------
[...] Metrics of PPA generation

======================
Instances
----------------------
[...] Connection between modules
[...] No connections for single modules

======================
AML: Example
----------------------
[...] Pseudo-representation of the module
\end{verbatim}
\end{small}

The pseudo-code underneath \textit{"AML: Example"} shows an abstract
representation of the module. %
The language used here is called \textit{"Abstract Model Language"}
and is based on the specification described as \href{https://www.eit.uni-kl.de/fileadmin/eis/PPA/designflow/AML-syntax.txt}{EBNF}.
We use this DSL as an intermediate format in our flow.

The file \textit{WalkThrough\_with\_error.h} contains a statement that is not part of the subset and the output of \textit{DeSCAM} reports the error. %
If a statement is found to be not part of the subset it is ignored for the abstract model. %
\newpage
\begin{small}
\begin{verbatim}
======================
Errors: Translation of Stmts for module Example
----------------------
- test.push_back(value)
  -E- push_back() is not a supported method!
\end{verbatim}
\end{small}
For example, using a \tsCODE{std::vector} on system level is
beneficial when it comes to executing the simulation on the CPU but it
is not transferable to a hardware system, because memory
implementations have a static and known size. %
Thus, the statement is reported by DeSCAM as not being part of the
subset. %
The designer must now make a decision, {\begin{itemize}
  \item whether or not the statement is important for the module behavior,
  \item whether the SystemC description has to be modified,
  \item or whether the statement can be safely ignored as it has no
    effect on the behavior (e.g., a \tsCODE{std::cout}). %
 \end{itemize}


\subsection{Step 2: Property Generation}
\label{sec:walk-through-step2}

Once the \SYSTEMCPPA{} code is stable and deemed ready for RTL
implementation, we can generate the initial set of abstract
properties using \DeSCAM{}. %
This is the second step of PDD in which we generate the properties with our tool \textit{DeSCAM}. 
A single SystemC module or a system of SystemC modules serves as the input to the tool. %
The tool parses the files and generates an abstract model that is used for the generation of the properties. % 
In order to generate the properties run \tsCODE{./DeSCAM path-to-file/WalkThrough.h -ITL}. %
The \tsCODE{-ITL} flag invokes property generation in ITL (\emph{InTerval Language}). %
ITL is a property specification language available for the OneSpin~\cite{web-onespin} property checker. %
We support it because the language is quite intuitive and apt to specifying the kind of properties used in PDD. %
Of course, also other property specification languages can be used in PDD. %
Currently, besides ITL,  \DeSCAM{} also support SystemVerilog Assertion (SVA)~\cite{2009-SystemVerilog}. %
For SVA  generation run the tool with the option \tsCODE{-SVA}. %
For a full list of possible commands run \tsCODE{./DeSCAM -help}. %

In this section, we will provide only a quick overview over the
generated abstract properties. %
For a more detailed explanation refer to Sec~\ref{sec:pdd}. In
order to link the abstract objects of the system level to a concrete
RTL implementation we introduce \emph{macros}. %
The RT designer refines these macros by replacing the \tsCODE{MACRO\_BODY} with concrete information from the RT design. 
For \tsCODE{b\_in} and \tsCODE{b\_out} the following macros are generated:

\begin{small}
\begin{verbatim}
-- SYNC AND NOTIFY SIGNALS
macro b_in_notify :boolean:= MACRO_BODY end macro; 
macro b_in_sync   :boolean:= MACRO_BODY end macro; 
macro b_out_notify:boolean:= MACRO_BODY end macro; 
macro b_out_sync  :boolean:= MACRO_BODY end macro; 
-- DP SIGNALS -- 
macro b_in_sig : int := MACRO_BODY end macro; 
macro b_out_sig : bool := MACRO_BODY end macro; 
\end{verbatim}
\end{small}

There are two types of macros -- synchronization macros for event signalling, and data path macros for transporting the message content. %
The synchronization macro names have the suffixes \tsCODE{sync} and  \tsCODE{notify}. %
Thes signals are of Boolean type and implement the four-phase handshake. %
The generated data path macro names have the suffix \tsCODE{sig}. %
The return type of these macros is the type of the transported message. %
The designer specifies, by refining these macros, how the abstract message is encoded at the RTL. %

The protocol for sending (receiving) a message with a four-phase handshake is:
\begin{itemize}
\item The writer (reader) set the outgoing notify signal to
  \textit{true}. %
\item The writer (reader) waits until the incoming sync signal
  evaluates to \textit{true}. %
\item The message is exchanged by writing (reading) the datapath
  signal. %
\item After message exchange, the writer (reader) unsets the notify signal
\item End of the transmission. %
\end{itemize}

At the system level the handshaking is implicitly implemented through
events and, thus, not visible to the user, whereas at the RTL there
has to be an explicit implementation of the handshaking. %
 
In our methodology, the generated properties capture the full I/O
behavior of the system-level design. %
The system-level thread is transformed into a special finite state
machine (\textit{FSM}). 
This FSM represents a \emph{path predicate abstraction}
(\textit{PPA})~\cite{2015-UrdahlStoffel.etal, 2014-UrdahlStoffel.etal,
  2016-UrdahlUdupi.etal} of the RTL design to be implemented. %
The states of the FSM represent the points of communication between
modules, and the transitions between states represent the computations
inside the module, as specified by the properties. %

Listing~\ref{lst:walk-through} contains three communication calls (line~17, 19, 20). %
For each of these calls, a state is created. 
These states are called \textit{important states}, because they
represent important control points of the hardware. %

When refining the property suite, the designer needs to fill in the
corresponding state macros in order to specify what conditions in the
RTL hardware represent the states. %
In order to do so it is allowed to use internal signals as well as outputs. %

\begin{small}
\begin{verbatim}
-- STATES -- 
macro run_0 : boolean :=  end macro;
macro run_1 : boolean :=  end macro;
macro run_2 : boolean :=  end macro;
\end{verbatim}
\end{small}

There is a directed edge between two important states if there exists an
execution path between the corresponding communication calls at the system level. %
Each edge translates into an interval property describing the actions taken in the RTL when moving from one important state to the next. %
The following property starts in the important state \tsCODE{run\_0},
which is the state after reset. %

\begin{small}
\begin{verbatim}
property run_0_read_0 is
dependencies: no_reset;
for timepoints:
  t_end = t+1; -- CHANGE HERE
assume: 
  at t: run_0;
  at t: (b_in_sig >= 11);
  at t: b_in_sync;
prove:
  at t_end: run_1;
  at t_end: b_out_sig=true;
  during[t+1,t_end]: b_in_notify=false;
  during[t+1,t_end-1]: b_out_notify=false;
  at t_end: b_out_notify = true;
end property;
\end{verbatim}
\end{small}

The property run\_0\_read\_0 describes the execution path between
line~17 and line~19 of Figure~\ref{lst:walk-through}. %

If the input is available ($b\_in\_sync == true$) and the input value
is greater or equal~11 the hardware moves to important state
\tsCODE{run\_1}. %
In \tsCODE{run\_1} the value of \tsCODE{b\_out\_sig} is set to
\textit{true} and the counterpart is notified by raising
\tsCODE{b\_out\_notify}. The property assumes that:
\begin{itemize}
\item The hardware is in state \tsCODE{run\_0},
\item a new input is available ($b\_in\_sync == true$),
\item and the value of the received message is greater or equal~11 (\tsCODE{b\_in\_sig}~$\geq$~11). 
\end{itemize} 
Then, it is proven that: 
\begin{itemize}
\item The hardware moves to important state \tsCODE{run\_1},
\item the value of \tsCODE{b\_out\_sig} is set to \textit{true},
\item and the outgoing message is offered by raising
  \tsCODE{b\_out\_notify}. %
\end{itemize}

The hardware remains in state \tsCODE{run\_1} until the recipient of
the message accepts the handshake and the message is passed. %

In case the module has to wait for the communication partner, a
\emph{wait} property is generated automatically for each blocking
communication that enforces the hardware to remain in the same
important state until the respective \tsCODE{sync} signal is active. %

\subsection{Step 3: Hardware Implementation}
\label{sec:walk-through-step3}

Now, the RT designer has the task to specify how the abstract
system-level objects are implemented by the RT-design by filling in the macros. %
For example, the macro \tsCODE{b\_in\_sig} represents the incoming value. %
The designer has  full freedom in how to implement a representation of this data. %
The designer maps the corresponding RTL signals to the property by providing a body for this macro. %

Let us suppose that the incoming message be transported by the input signal \tsCODE{value\_in} and, thus, the macro is refined into
\begin{small}
 \begin{verbatim} 
macro b_in_sig : int := 
  RT_design/value_in 
end macro; 
\end{verbatim}
\end{small}
It could also be that a message is transported by two different RT signals and the resulting value is the sum of those two signals. %
The designer would, accordingly, provide this information in the macro body:
\begin{small}
  \begin{verbatim} 
macro b_in_sig : int := 
  RT_design/input1 + RT_design/input2  
end macro; 
\end{verbatim}

\end{small}

In case there exists no previous implementation, the RT designer has
the option to generate a VHDL template by running \tsCODE{./DeSCAM
  <path-to-file>/WalkThrough.h -VHDL}. %
This template contains a package with all used data types and a
bare-bone VHDL implementation matching the properties. %
The template is not a synthesized design and thus doesn't contain any
behavioral components. %

In order to start the hardware design process, load your VHDL template
or empty design and the accompanying properties with the property
checker. %
The first property to prove is the reset property. %
The design process continues with implementing all operations starting
in the important state after reset. %
The RT designer refines the important state macros for the currently
implemented operations during the design process. %

%%% Local Variables: 
%%% mode: latex
%%% TeX-master: "binder"
%%% End: 

